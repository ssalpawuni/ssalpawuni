\documentclass[12pt,a4paper, fleq]{article}

\usepackage[utf8]{inputenc}
\usepackage{amsmath}
\usepackage{amsfonts}
\usepackage{amssymb}
\usepackage{graphicx}
\usepackage{enumerate}
\usepackage{booktabs}
\usepackage{adjustbox}
\usepackage{relsize}
\usepackage{float}
\usepackage[sc]{mathpazo} 
\usepackage{microtype}

% Setting margins of the document ==========================================================
\usepackage{geometry} % Required for adjusting page dimensions and margins

\geometry{
	paper=a4paper, % Paper size
	top=2.5cm, % Top margin
	bottom=3cm, % Bottom margin
	left=2.5cm, % Left margin
	right=2.5cm, % Right margin
	headheight=14pt, % Header height
	footskip=1.5cm, % Space from the bottom margin to the baseline of the footer
	headsep=1.2cm, % Space from the top margin to the baseline of the header
}
% PGF plots
\usepackage{pgfplots, pgfplotstable}

\begin{document}

%====== creating the title of the document===============
\title{\textcolor{blue}{Tutorial 3 (Abubakari Sumaila Salpawuni)}}\vskip -3mm

\date{\textcolor{blue}{\today}}
\maketitle
%=========================================

%++++++++++++++++
% Problem 1
%++++++++++++++++
\subsection*{$Q_{1}$ -- solution}

\begin{itemize}
\item[1.(a)] % Question 1.a
Given $X \sim U(0, 1)$. Define $g(x)$ as the area the square such that;
\begin{equation*}
\begin{split}
\mbox{Area} & = g(x) = X^2\\
E\big(g(X)\big) &= \int_0^1 g(x)f_X(x)dx\\
E(X^2) & = \int_0^1 x^2 dx = \frac13x^3\bigg\lvert_0^1\\
& = \frac 13 \mbox{sq. units}
\end{split}
\end{equation*}

\item[1.(b)] % Question 1.b
If $X_i \sim Poi(\lambda_i)$, then it moment generating function, {\it m.g.f}, is $M_X(t) = e^{[\lambda_i(e^t - 1)]}$. Since the variables are independent, we have;
\begin{equation*}
\begin{split}
M_Y & = M_{X_1}(t) \cdot M_{X_2}(t)\ldots \cdot M_{X_n}(t)\\
& = e^{[\lambda_1 (e^t - 1)]} \cdot e^{[\lambda_2 (e^t - 1)]} \cdot \ldots \cdot e^{[\lambda_n (e^t - 1)]}\\
& = e^{[(\lambda_1 + \lambda_2 + \ldots + \lambda_n)(e^t - 1)]}\\
& = e^{[\sum_{i=1}^n\lambda_i{(e^t - 1)]}}
\end{split}
\end{equation*}
$\therefore  \sum_{i=1}^n X_i \sim \mbox{Poi} (\sum_i^n\lambda_i)$

\item[1.(c)] % Question 1.c
Given a density function, one way to solve this is to use  the {\it method of transformation}.

 Let $y = g(x) = \frac1x$. The density of $g(y)$ can be computed as;
$$g(y) = \left \lvert \frac{dx}{dy} \right \rvert \cdot f(W(y))$$ where $W(y)$ is the inverse function of  g(x).

\begin{equation*}
\begin{split}
y & = \frac1x \implies x  = \frac1y\\
\left \lvert \frac{dx}{dy} \right \rvert & = -\frac{1}{y^2} = \frac{1}{y^2}
\end{split}
\end{equation*}

\begin{equation*}
\begin{split}
g(x) &= \frac{1}{y^2} \times \frac{1}{(n-1)!\beta^n} \left(\frac{1}{y}\right)^{n-1}e^{-\frac{1}{y\beta}}\\
 & =  \frac{1}{(n-1)!\beta^n} \left(\frac{1}{y}\right)^{(n+2) - 1}e^{-\frac{1}{y\beta}}\\
\therefore g\left(\frac1x\right) & =  \mbox{Gamma}(n+2, \beta)
\end{split}
\end{equation*}

\end{itemize}



%++++++++++++++++
% Problem 2
%++++++++++++++++
\subsection*{$Q_{2}$ -- solution}
This is clearly a Bayesian problem. Define $D$ as an event that a subject has the disease; $T^+$ be event that a subject's test result returns positive, and the event that a subject's test result $T^-$ returns negative. Thus far, we proceed with the following pieces of information;

\begin{equation*}
\begin{split}
P(D) &= 0.015 \quad \mbox{(prevalence)}\\
P(T^+\vert D) &= 0.97 \quad \mbox{(sensitivity)}\\
P(T^-\vert D^\prime) &= 0.95 \quad \mbox{(specificity)}
\end{split}
\end{equation*}

\begin{itemize}
\item[2.(a)] % Question 2.(a)
\begin{itemize}
\item[i.]
We partition $T$ into $T^+$ and $T^-$. Thus, 
\begin{equation*}
\begin{split}
P(T^+|D) & = P(D)P(T^+\vert D) + P(D^\prime)P(T^+\vert D^\prime)\\
& = 0.015 \times 0.97 + (1-0.015) \times (1-0.95)\\
& = 0.905975
\end{split}
\end{equation*}

\item[ii.]
We're interested in $P(D\vert T^+)$, the probability you have the disease given
you tested positive to the disease. Using the Bayesian formulation, we proceed as follows:
\begin{equation*}
\begin{split}
P(D\vert T^+) &= \frac{P(D)P(T^+\vert D)}{P(D)P(T^+\vert D) + P(D^\prime)P(T^+\vert D^\prime)}\\[2mm]
&= \frac{0.015 \times 0.97}{0.905975}\\[2mm]
&= 0.0161 (\mathbf{1.61\%})
\end{split}
\end{equation*}
\end{itemize}

\item[2.(b)] % Question 2.(b)
$P(A) = 0.40, P(B) = 0.10, P(C) = 0.50$\\
Let $D$ be the {\it event} that a defective product is produced\\
$P(D|A) = 0.02$\\
$P(D|B) = 0.03$ \\
$P(D|C) = 0.04$ 

\begin{itemize}
\item[i.]
\begin{equation*}
\begin{split}
P(D) &= P(A)P(D|A) + P(B)P(D|B) + P(C)P(D|C)\\
& = 0.40(0.02) + 0.10(0.03) + 0.50(0.04)\\
& = 0.031
\end{split}
\end{equation*}

\item[ii.]
$\alpha.$ $P(A|D) = \frac{P(A)P(D|A)}{P(D)} = \frac{0.40\times 0.02}{0.031} = \frac{8}{31}$

$\beta.$ $P(B|D) = \frac{P(B)P(D|A)}{P(D)} = \frac{0.10\times 0.03}{0.031} = \frac{3}{31}$

$\gamma.$ $P(C|D) = \frac{P(C)P(D|A)}{P(D)} = \frac{0.50\times 0.04}{0.031} = \frac{20}{31}$
\end{itemize}

\item[3.(c)] % Question 2.(c)
Using Matrix approach, Let $\hat{y} = X\beta + \epsilon$ such that;
$$\beta = (X^\prime X)^{-1}X^\prime y$$

\begin{equation*}
\begin{split}
X^\prime X & =
\begin{pmatrix}
1 & 1 & \ldots & 1\\
5 & 6 & \ldots & 10\\
1 & 1 & \ldots & 6
\end{pmatrix}
\begin{pmatrix}
1 & 5 & 1\\
1 & 6 & 1\\
\vdots & \vdots & \vdots\\
1 & 10 & 6 
\end{pmatrix}\\
& = \begin{pmatrix}
10 & 70 & 40\\
70 & 514 & 298\\
40 & 298 & 194
\end{pmatrix}
\end{split}
\end{equation*}
\begin{equation*}
\begin{split}
(X^\prime X)^{-1} =
\begin{pmatrix}
2.2179 & -0.3374 & -0.0610\\
-0.3374 & 0.0691 & −0.0366\\
-0.0610 & −0.0366 & 0.0488
\end{pmatrix}
\end{split}
\end{equation*}
\begin{equation*}
\begin{split}
X^\prime y =
\begin{pmatrix}
60\\
449\\
264
\end{pmatrix}
\end{split}
\end{equation*}
\begin{equation*}
\begin{split}
\hat{\beta} & =
\begin{pmatrix}
2.2179 & -0.3374 & -0.0610\\
-0.3374 & 0.0691 & −0.0366\\
-0.0610 & −0.0366 & 0.0488
\end{pmatrix}
\begin{pmatrix}
60\\
449\\
264
\end{pmatrix}\\[2mm]
& = \begin{pmatrix}
-2.3146\\
1.1195\\
0.1098
\end{pmatrix}\\[3mm]
\therefore \hat{\beta} & = -2.3146 + 1.1195X_1 + 0.1098X_2
\end{split}
\end{equation*}
\end{itemize}

%++++++++++++++++
% Problem 3
%++++++++++++++++
\subsection*{$Q_{3}$ -- solution}
\begin{itemize}
\item[3.(a)]
\begin{itemize}
\item[i.]
\begin{equation*}
\begin{split}
\int_{-\infty}^\infty xf(x)dx & = \int_{-1}^0 \frac{2x|x|}{5}dx + \int_{0}^2 \frac{2x|x|}{5}dx\\[2mm]
E(X) & = -\frac25 \int_{-1}^0 x^2dx + \frac25 \int_{0}^2 x^2dx\\[2mm]
& = -\frac{2}{15}x^3\bigg\lvert_{-1}^0 + \frac{2}{15}x^3\bigg\lvert_0^2\\[2mm]
& = -\frac{2}{15} + \frac{16}{15}\\
& = \frac{14}{15}.
\end{split}
\end{equation*}

\item[ii.]
\begin{equation*}
\begin{split}
\mbox{Var(X)} & = E(X^2) - [E(X)]^2 \\[2mm]
& = -\int_{-1}^0 \frac{2x^2|x|}{5}dx + \int_{0}^2 \frac{2x^2|x|}{5}dx - \left(\frac{14}{15}\right)^2\\[2mm]
& = -\frac{1}{10}x^4\bigg\lvert_{-1}^0 + \frac{1}{10}x^4\bigg\lvert_0^2 - \left(\frac{14}{15}\right)^2\\[2mm]
& = -\frac{1}{10} + \frac{16}{10} - \left(\frac{14}{15}\right)^2 = \frac{17}{10}-\frac{196}{225}\\
& = \frac{95}{102}.
\end{split}
\end{equation*}
\end{itemize}

\item[3.(b)]
\begin{equation*}
Y\sim U(-1, 8) = 
\begin{cases}
\frac19, & -1 < x < 8\\
\\
0, & \mbox{otherwise}
\end{cases}
\end{equation*}

For the equation $2x^2 + 4Yx + 3Y+2 = 0$ to have real roots, the *discriminant* must be non-negative, thus;

\begin{equation*}
\begin{split}
b^2 - 4ac \ge 0 & \implies (4Y)^2 - 4(2)(3Y+2) \ge 0\\
& \implies 16Y^2-24Y + 16 \ge 0\\
& \implies 2Y^2 -8Y + 2 \ge 0\\
&\implies Y \ge 2\quad \mbox{or}\quad Y \le -\frac12
\end{split}
\end{equation*}
Using the probability density function above, and the idea of mutually exclusive events, we have;
\begin{equation*}
\begin{split}
P(Y \le -\frac12 \cup Y \ge 2)  & = P(Y \le -\frac12) + P(Y \ge 2)\\[2mm]
& = \int_{-1}^{\frac12}f_Y(y)dy + \int_2^8 f_Y(y)dy\\[2mm]
& = \frac19(\frac12 + 6)\\[2mm]
& = \frac{13}{18}.
\end{split}
\end{equation*}

\item[3.(c)]
Note that the {\it moment generating function} for the {\it Exponential distribution} is given as $M_X(t) = \frac{\lambda}{\lambda - t}, \quad t < \lambda$. Comparing this with the above function, we may infer that;
\begin{equation*}
f(x) = 
\begin{cases}
\frac12 e^{-\frac12\lambda t}, & t < \lambda \\
\\
0, & \mbox{otherwise}
\end{cases}
\end{equation*}

\begin{itemize}
\item[i.]
The {\it cumulative density function} for the Exponential distribution is thus;
$$1 - F(x) = e^{-\lambda t},\quad t < \lambda$$
$\therefore P(X > \mbox{In4}) = e^{-\frac12\mbox{In4}} = \frac12.$

\item[ii.]
\begin{equation*}
\begin{split}
E(X) & = \frac{d}{dx}M_X(t)\bigg\lvert_{t = 0}\\
& = \frac{0.5}{(0.5 - t)^2}\bigg\lvert_{t=0}\\
& = 2\\[4mm]
E(X^2) & = \frac{d^2}{dx^2}M_X(t)\bigg\lvert_{t = 0}\\
& = \frac{2(0.5)}{(0.5 - t)^3}\bigg\lvert_{t= 0}\\
& = 8\\[4mm]
\mbox{Var(X)} & = E(X^2) - [E(X)]^2\\
& = 8 - 2^2 = 4\\
\therefore \mbox{Var(X)} & = 4.
\end{split}
\end{equation*}

\end{itemize}
\end{itemize}

\end{document}